
\newcommand{\sepline}{%
	\begin{center}
	    \rule{\linewidth}{0.4pt}
	\end{center}
}

\section{Csomóponti módszer}
Ezen módszerben csomópontonként vizsgálom az ismeretlen \textcolor{cyan}{rúderőket}, amiket pozitívnak (tehát húzott rudakat) feltételezek: ezért kifele mutatónak ábrázolom őket.

A használt szögeket triviális trigonometriai összefüggésekkel számoltam ki illetve innentől az egész dokumentumban a pipák azt jelentik hogy az eddig kiszámolt értékeink egyeznek az új értékekkel (ezzel is ellenőrizve számolásaimat).

% A
\begin{minipage}{0.4\textwidth}
	\centering
	\begin{tikzpicture}
		\draw circle (1) node[midway, font=\huge] {A};
		\draw[line width = .8mm, arrows = {-Stealth}, cyan] (1, 0) -- ++(2, 0) node[midway, above] {$N_2$};
		\draw[line width = .8mm, arrows = {-Stealth}, green] (-3, 0) -- ++(2, 0) node[midway, above] {$A_x$};
		\draw[line width = .8mm, arrows = {-Stealth}, green] (0, -3) -- ++(0, 2) node[midway, left] {$A_y$};
		\draw[line width = .8mm, arrows = {-Stealth}, cyan] (-\alfa:1) -- ++(-\alfa:2) node[midway, left] {$N_1$};
		\draw[black] (2, 0) arc[start angle=0, end angle=-\alfa, radius=2] node[midway, anchor=south east, font=\huge] {$\alpha$};

	\end{tikzpicture}
\end{minipage}
\begin{minipage}{0.5\textwidth}
	\begin{align*}
		&\sum{\vec{F}_x} := 0 = A_x + N_2 + {N_1}_x\\
		&\sum{\vec{F}_y} := 0 = A_y - {N_1}_y
	\end{align*}
	\begin{center}
		$\tan \alpha = \frac{c}{a} \Rightarrow \alpha = \pnum{\alfa}$\textdegree
	\end{center}
	\begin{align*}
		&{N_1}_y = A_y = &\pnum{\Nay} [\si{kN}] \\
		&\mathbf{N_1} = \frac{{N_1}_y}{\sin \alpha} = &\mathbf{\pnum{\Na}} [\si{kN}] \\
		&{N_1}_x = N_1 \times \cos \alpha = &\pnum{\Nax} [\si{kN}]
	\end{align*}
\end{minipage}


\sepline

% B
\begin{minipage}{0.4\textwidth}
	\centering
	\begin{tikzpicture}
		\draw circle (1) node[midway, font=\huge] {B};
		\draw[line width = .8mm, arrows = {-Stealth}, cyan] (-1, 0) -- ++(-2, 0) node[midway, above] {$N_8$};
		\draw[line width = .8mm, arrows = {-Stealth}, green] (1, 0) -- ++(2, 0) node[midway, above] {$B_x$};
		\draw[line width = .8mm, arrows = {-Stealth}, green] (0, -3) -- ++(0, 2) node[midway, left] {$B_y$};
		\draw[line width = .8mm, arrows = {-Stealth}, cyan] (0, 1) -- ++(0, 2) node[midway, left] {$N_{10}$};

	\end{tikzpicture}
\end{minipage}
\begin{minipage}{0.5\textwidth}
	\begin{align*}
		&\sum{\vec{F}_x} := 0 = B_x - N_8 \\
		&\sum{\vec{F}_y} := 0 = B_y + N_{10}
	\end{align*}
	\begin{align*}
		\mathbf{N_8} &= B_x = &\mathbf{\pnum{\Nh}} [\si{kN}] \\
		\mathbf{N_{10}} &= -B_y = &\mathbf{\pnum{\Nj}} [\si{kN}]
	\end{align*}
\end{minipage}

\sepline

% C
\begin{minipage}{0.4\textwidth}
	\centering
	\begin{tikzpicture}
		\draw circle (1) node[midway, font=\huge] {C};
		\draw[line width = .8mm, arrows = {-Stealth}, cyan] (180-\alfa:1) -- ++(180-\alfa:2) node[midway, right] {$N_5$};
		\draw[black] (-2, 0) arc[start angle=180, end angle=180-\alfa, radius=2] node[midway, anchor=north west, font=\huge] {$\alpha$};
		\draw[line width = .8mm, arrows = {-Stealth}, cyan] (-1, 0) -- ++(-2, 0) node[midway, below] {$N_6$};
		\draw[line width = .8mm, arrows = {-Stealth}, cyan] (0, 1) -- ++(0, 2) node[midway, left] {$N_7$};
		\draw[line width = .8mm, arrows = {-Stealth}, cyan] (1, 0) -- ++(2, 0) node[midway, below] {$N_8$};
		\draw[line width = .8mm, arrows = {-Stealth}, cyan] (90-\Beta:1) -- ++(90-\Beta:2) node[midway, anchor=south east] {$N_9$};
		\draw[black] (2.5, 0) arc[start angle=0, end angle=90-\Beta, radius=2.5] node[midway, anchor=north east] {$\ang{90}-\beta$};
	\end{tikzpicture}
\end{minipage}
\begin{minipage}{0.5\textwidth}
	\begin{align*}
		&\sum{\vec{F}_x} := 0 = N_8 - N_6 + {N_9}_x - {N_5}_x \\
		&\sum{\vec{F}_y} := 0 = N_7 + {N_5}_y + {N_9}_y 
	\end{align*}
	\begin{align*}
		&\mathbf{N_6} = N_8 + {N_9}_x - {N_5}_x = &\mathbf{\pnum{\Nf}} [\si{kN}] \\
		&\mathbf{N_7} = -{N_5}_y - {N_9}_y = &\mathbf{\pnum{\Ng}} [\si{kN}]
	\end{align*}
\end{minipage}

\sepline

% D
\begin{minipage}{0.4\textwidth}
	\centering
	\begin{tikzpicture}
		\draw circle (1) node[midway, font=\huge] {D};
		\draw[line width = .8mm, arrows = {-Stealth}, cyan] (1, 0) -- ++(2, 0) node[midway, above] {$N_4$};
		\draw[line width = .8mm, arrows = {-Stealth}, cyan] (-1, 0) -- ++(-2, 0) node[midway, above] {$N_2$};
		\draw[line width = .8mm, arrows = {-Stealth}, cyan] (0, -1) -- ++(0, -2) node[midway, left] {$N_3$};
		\draw[line width = .8mm, arrows = {-Stealth}, cyan] (-\alfa:1) -- ++(-\alfa:2) node[midway, left] {$N_5$};
		\draw[black] (2, 0) arc[start angle=0, end angle=-\alfa, radius=2] node[midway, anchor=south east, font=\huge] {$\alpha$};

	\end{tikzpicture}
\end{minipage}
\begin{minipage}{0.5\textwidth}
	\begin{align*}
		&\sum{\vec{F}_x} := 0 = N_4 - N_2 + {N_5}_x \\
		&\sum{\vec{F}_y} := 0 = -N_3 - {N_5}_y 
	\end{align*}
	\begin{align*}
		{N_5}_x &= N_2 - N_4 = &\pnum{\Nex} [\si{kN}] \\
		{N_5}_y &= -N_3 = &\pnum{\Ney} [\si{kN}] \\
		\mathbf{N_5} &= \sqrt{{{N_5}_x}^2 + {{N_5}_y}^2} = &\mathbf{\pnum{\Ne}} [\si{kN}] \\
	\end{align*}
\end{minipage}

\sepline

% E
\begin{minipage}{0.4\textwidth}
	\centering
	\begin{tikzpicture}
		\draw circle (1) node[midway, font=\huge] {E};
		\draw[line width = .8mm, arrows = {-Stealth}, cyan] (180-\alfa:1) -- ++(180-\alfa:2) node[midway, right] {$N_1$};
		\draw[black] (-2, 0) arc[start angle=180, end angle=180-\alfa, radius=2] node[midway, anchor=north west, font=\huge] {$\alpha$};
		\draw (-1, 0) -- ++(-2, 0);
		\draw[line width = .8mm, arrows = {-Stealth}, cyan] (0, 1) -- ++(0, 2) node[midway, right] {$N_3$};
		\draw[line width = .8mm, arrows = {-Stealth}, cyan] (1, 0) -- ++(2, 0) node[midway, below] {$N_8$};
		\draw[line width = .8mm, arrows = {-Stealth}, blue] (0, -1) -- ++(0, -2) node[midway, left] {$F_1$};
	\end{tikzpicture}
\end{minipage}
\begin{minipage}{0.5\textwidth}
	\begin{align*}
		&\sum{\vec{F}_x} := 0 = N_6 - {N_1}_x \\
		&\sum{\vec{F}_y} := 0 = N_3 - F_1 + {N_1}_y
	\end{align*}
	\begin{align*}
		&N_6 = {N_1}_x = &\pnum{\Nf} &[\si{kN}] \checkmark \\
		&\mathbf{N_3} = F_1 - {N_1}_y = &\mathbf{\pnum{\Nc}} &[\si{kN}] \\
	\end{align*}
\end{minipage}

\sepline

% F
\begin{minipage}{0.4\textwidth}
	\centering
	\begin{tikzpicture}
		\draw circle (1) node[midway, font=\huge] {F};
		\draw[line width = .8mm, arrows = {-Stealth}, cyan] (-1, 0) -- ++(-2, 0) node[midway, below] {$N_4$};
		\draw[line width = .8mm, arrows = {-Stealth}, blue] (1, 0) -- ++(2, 0) node[midway, below] {$F_3$};
		\draw[line width = .8mm, arrows = {-Stealth}, cyan] (0, -1) -- ++(0, -2) node[midway, left] {$N_7$};
	\end{tikzpicture}
\end{minipage}
\begin{minipage}{0.5\textwidth}
	\begin{align*}
		&\sum{\vec{F}_x} := 0 = F_3 - N_4 \\
		&\sum{\vec{F}_y} := 0 = -N_7
	\end{align*}
	\begin{align*}
		&\mathbf{N_4} = F_3 = &\mathbf{\pnum{\Nd}} &[\si{kN}] \\
		&N_7 = &\pnum{\Ng} &[\si{kN}] \checkmark \\
	\end{align*}
\end{minipage}

\sepline

% G
\begin{minipage}{0.4\textwidth}
	\centering
	\begin{tikzpicture}
		\draw circle (1) node[midway, font=\huge] {G};
		\draw[line width = .8mm, arrows = {-Stealth}, cyan] (180+\Beta:1) -- ++(180+\Beta:2) node[midway, anchor=south east] {$N_9$};
		\draw (180+\Beta:2) arc[start angle=180+\Beta, end angle=270, radius=2] node[midway, above, font=\LARGE] {$\beta$};
		\draw[line width = .8mm, arrows = {-Stealth}, cyan] (0, -1) -- ++(0, -2) node[midway, right] {$N_{10}$};
		\draw[line width = .8mm, arrows = {-Stealth}, blue] (3, 0) -- ++(-2, 0) node[midway, below] {$F_2$};
	\end{tikzpicture}
\end{minipage}
\begin{minipage}{0.5\textwidth}
	\begin{align*}
		&\sum{\vec{F}_x} := 0 = -F_2 - {N_9}_x \\
		&\sum{\vec{F}_y} := 0 = -N_{10} - {N_9}_y 
	\end{align*}
	\begin{center}
		$\tan \beta = \frac{b}{c} \Rightarrow \beta = \pnum{\Beta}$\textdegree
	\end{center}
	\begin{align*}
		{N_9}_x &= -F_2 = &\pnum{\Nix} [\si{kN}] \\
		\mathbf{N_9} &= \frac{{N_9}_x}{\sin\beta} = &\mathbf{\pnum{\Ni}} [\si{kN}] \\
		{N_9}_y &= N_9 \times \cos\beta = &\pnum{\Niy} [\si{kN}] \\
		\mathbf{N_{10}} &= -{N_9}_y = &\mathbf{\pnum{\Nj}} [\si{kN}]
	\end{align*}
\end{minipage}
