
\section{Részekre bontás elve}

\subsection{Szabadtest-ábra (SZTÁ)}

\begin{center}
	\begin{tikzpicture}
		\struccoordsys
		\strucbend
	\end{tikzpicture}
	\begin{tikzpicture}
		\strucframe
		\strucforces
		\strucrestrainforces
		\foreach \point in {C} {
			\fill[thick, orange] (\point) circle[radius=4pt];
		}
	\end{tikzpicture}
\end{center}
A két kényszerben $(\textcolor{green}{\mathbf{A}}, \textcolor{green}{\mathbf{B}})$ ébredő reakcióerőt pozitívként rajzoltam fel.

\break

\subsection{Részek vizsgálata, egyensúlyi egyenletek}

A \textcolor{orange}{$\mathbf{C}$} pontban kettévágva a rácsszerkezetet részenként vizsgálhatom (így ezen pont mindkét ábrának része). Az ebben a pontban ébredő belső reakcióerőket a két részen ellentétesen veszem fel \textbf{Newton III. törvénye} (hatás-ellenhatás) miatt.

\begin{multicols}{2}

\subsubsection{}
\begin{center}
	\begin{tikzpicture}
		\strucframepartone
		\strucforcespartone
		\strucrestrainforcespartone
		\foreach \point in {C} {
			\fill[thick, orange] (\point) circle[radius=4pt];
		}
		\struccutforcespartone
	\end{tikzpicture}
\end{center}
\begin{align*}
	&\sum{\vec{F}_x} := 0 = A_x + F_3 + C_x \\
	&\sum{\vec{F}_y} := 0 = A_y - F_1 + C_y \\
	&\sum{\vec{M}_C} := 0 = -A_x \times c - A_y \times 2a + F_1 \times a -F_3 \times c
\end{align*}

\begin{align*}
	&A_y = F_1 - C_y &= \pgfmathprintnumber[fixed]{\Ay} [\si{kN}] \\
	&A_x = \frac{-A_y \times 2a+F_1 \times a - F_3 \times c}{c} &= \pgfmathprintnumber[fixed]{\Ax} [\si{kN}] \\
	&C_x = -A_x - F_3 &= \pgfmathprintnumber[fixed]{\Cx} [\si{kN}]
\end{align*}

\subsubsection{}
\begin{center}
	\begin{tikzpicture}
		\strucframeparttwo
		\strucforcesparttwo
		\strucrestrainforcesparttwo
		\foreach \point in {C} {
			\fill[thick, orange] (\point) circle[radius=4pt];
		}
		\struccutforcesparttwo
	\end{tikzpicture}
\end{center}
\begin{align*}
	&\sum{\vec{F}_x} := 0 = B_x - F_2 - C_x \\
	&\sum{\vec{F}_y} := 0 = B_y - C_y \\
	&\sum{\vec{M}_C} := 0 = F_2 \times c + B_y \times b
\end{align*}

\begin{align*}
	&B_y = -F_2 \frac{c}{b} &= \pgfmathprintnumber[fixed]{\By} [\si{kN}] \\
	&C_y = B_y &= \pgfmathprintnumber[fixed]{\Cy} [\si{kN}] \\
	&B_x = C_x + F_2 &= \pgfmathprintnumber[fixed]{\Bx} [\si{kN}]
\end{align*}

\end{multicols}

Mivel 6 ismeretlenünk és egyenletünk van, a reakció-erőrendszer megoldható és a keresett reakcióerők kifejezhetőek.

\begin{center}
	\fbox{
	    $
		\begin{aligned}
		    A &= \begin{bmatrix}
			    \pgfmathprintnumber[fixed]{\Ax} \\
			    \pgfmathprintnumber[fixed]{\Ay}
		    \end{bmatrix} &[\si{kN}]\\
		    B &= \begin{bmatrix}
			    \pgfmathprintnumber[fixed]{\Bx} \\
			    \pgfmathprintnumber[fixed]{\By}
		    \end{bmatrix} &[\si{kN}]
		\end{aligned}
	    $
	}
\end{center}
