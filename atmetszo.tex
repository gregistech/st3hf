
\section{Átmetsző módszer}
Ezen módszerrel \textcolor{orange}{két külön részre vágással} vizsgálhatom a szerkezetet. Arra kell figyelni hogy \textbf{maximum 3 nem párhuzamos rudat} "metszhetünk át".


\subsection{SZTÁ}
\begin{center}
	\begin{tikzpicture}
		\struccoordsys
		\strucbend
	\end{tikzpicture}
	\begin{tikzpicture}
		\strucframe
		\strucforces
		\strucrestrainforces
		\draw[loosely dashed, orange, line width=.8mm] ($ (D) !0.5! (F) $)+(-.5,2) -- ($ ($ (E) !0.5! (C) $) + (-.5, -2)$);
	\end{tikzpicture}
\end{center}

\break

\subsection{Megmaradt rúderők}
Mivel a keresett rúderők már az egyik rész egyensúlyi egyenleteiből is kiszámolhatóak, elég azt lerajzolnunk.

\begin{center}
	\begin{tikzpicture}
		\coordinate (A) at (0, \cs);
		\coordinate (D) at (\as, \cs);
		\coordinate (E) at (\as, 0);

		\draw[line width = 0.1mm] (A)+(-1,0) -- (A);
		\draw[line width = 0.1mm] (-1,0) -- (E);
		\draw[line width = 0.1mm] (A)++(0,2) -- ++(0,-\cs-4);
		\draw[line width = 0.1mm] (D)++(0,2) -- ++(0,-\cs-4);

		\draw[line width = 0.2mm, arrows = {Latex-Latex}] (-1,0) -- ++(0,\cs) node[midway, left] {c};
		\draw[line width = 0.2mm, arrows = {Latex-Latex}] (0,-1) -- ++(\as,0) node[midway, below] {a};

		\pgfmathsetmacro{\framewidth}{.7mm}
		\draw[line width = \framewidth, arrows = {-}] (A) -- (E) node[midway, anchor=north east] {1};
		\draw[line width = \framewidth, arrows = {-}] (A) -- (D) node[midway, above] {2};
		\draw[line width = \framewidth, arrows = {-}] (D) -- (E) node[midway, left] {3};

		\foreach \point/\name in {A, D} {
			\fill[white] (\point) circle[radius=2.5pt];
			\draw[thick] (\point) circle[radius=2pt] node[anchor=south west] {\name};
		}
		\foreach \point/\name in {E} {
			\fill[white] (\point) circle[radius=2.5pt];
			\draw[thick] (\point) circle[radius=2pt] node[anchor=south west] {\name};
		}

		\draw[line width = .8mm, arrows = {-Stealth}, blue] (E) -- ++(0, -\Fasp) node[right] {$F_1$};
		\draw[line width = .8mm, arrows = {-Stealth}, cyan] (D) -- ++(3, 0) node[midway, above] {$N_4$};

		\draw[line width = .8mm, arrows = {-Stealth}, cyan] (D) -- ++(-\alfa:3) node[midway, left] {$N_5$};
		\draw[black] (D) ++(2,0) arc[start angle=0, end angle=-\alfa, radius=2] node[midway, anchor=south east, font=\huge] {$\alpha$};

		\draw[line width = .8mm, arrows = {-Stealth}, cyan] (E) -- ++(3, 0) node[midway, above] {$N_6$};

	\draw[line width = .8mm, arrows = {-Stealth}, green] (A)++(-2, 0) -- (A) node[midway, above] {$A_x$};
	\draw[line width = .8mm, arrows = {-Stealth}, green] (A)++(0, -2) -- (A) node[midway, left] {$A_y$};
	\end{tikzpicture}
\end{center}

\begin{align*}
	&\sum{\vec{F}_x} := 0 = A_x + N_4 + {N_5}_x + N_6 \\
	&\sum{\vec{F}_y} := 0 = A_y - {N_5}_y - F_1 \\
	&\sum{\vec{M}_D} := 0 = A_y \times a + N_6 \times c \\
\end{align*}

\begin{align*}
	{N_5}_y &= A_y - F_2 = \pnum{\Ney} \checkmark \\
	\mathbf{N_5} &= \frac{{N_5}_y}{\sin\alpha} = \mathbf{\pnum{\Ne}} \checkmark \\
\end{align*}

\begin{align*}
	{N_5}_x &= N_5 \times \cos\alpha = \pnum{\Nex} \checkmark \\
	\mathbf{N_4} &= -A_x - {N_5}_x - N_6 = \mathbf{\pnum{\Nd}} \checkmark \\
	\mathbf{N_6} &= \frac{A_y \times a}{c} = \mathbf{\pnum{\Nf}} \checkmark \\
\end{align*}

Az értékeink egyeznek az előző feladattal, ezzel ellenőrizve munkámat.
